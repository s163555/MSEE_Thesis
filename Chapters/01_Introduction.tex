\chapter{Introduction} \label{cha:introduction} %\thispagestyle{main}
The progress of diagnostic imaging has advanced significantly during the \nth{20} century. As the cost of high-speed computational systems has grown increasingly accessible, so has the use of medical imaging become prominent. Potentially millions of people have been spared painful exploratory surgery through non-invasive diagnostic imaging. And thus, lives can be saved by early diagnosis and intervention through medical imaging. Advancements in scientific visualisation have in turn generated more complex datasets of increased size and quality. The four major technologies used are \gls{us}, X-ray, \gls{ct}, and \gls{mri}. Each technology has distinct advantages and disadvantages in biomedical imaging, and thus each is still relevant for modern medicine. \Cref{tab:imaging_modalities} contains a comparison and summary of the various fundamental diagnostic imaging modalities.

\begin{table}[ht]
	\centering
	\begin{talltblr}[
	caption = {Comparison of medical imaging modalities \cite{Szabo_UltrasoundBook_2}},
	entry = {Comparison of medical imaging modalities},
	label = {tab:imaging_modalities},
	note{a} = {Frequency and axially dependent.},
	note{b} = {Frequency dependent.},
	note{c} = {Fluoroscopy limited.},
	note{$\dag$} = {Typical: 45 minutes, fastest: Real-time (\glsxtrshort{low-res}).},
	]{
		%colspec = {Q[l,t]Q[l,t]Q[l,t]Q[l,t]Q[l,t]},
		%colspec = {Q[jQQQQ},
		row{1} = {guard, m, font=\small\bfseries},
	}
	\toprule
	\textbf{Modality} & \textbf{Ultrasound} & \textbf{X-ray} & \textbf{CT} & \textbf{MRI} \\ \midrule
	Topic             & {Longitudinal,\\shear,\\mechanical\\properties} & {Mean X-ray\\tissue\\absorption} & {Local tissue\\X-ray absorbtion} & {Biochemistry \\(\textit{T1} and \textit{T2})}    \\
	Access            & {Small\\windows\\adequate} & {2 sides\\needed} & {Circumferential\\around\\body} & {Circumferential\\around\\body} \\
	{Spatial\\resolution} & {\qty{0.2}{\milli\meter} to\\\qty{3}{\milli\meter}\TblrNote{a}} & $\sim \qty{1}{\milli \meter}$ & $\sim \qty{1}{\milli \meter}$ & $\sim \qty{1}{\milli \meter}$ \\
	Penetration     & {\qty{3}{\centi\meter} to\\\qty{25}{\centi\meter}\TblrNote{b}} & Excellent  & Excellent & Excellent \\
	Safety          & Excellent & {Ionizing\\radiation}      & {Ionizing\\radiation} & Very good \\
	Speed           & Real-time & Minutes & 20 minutes & Varies\TblrNote{$\dag$} \\
	Cost            & \$ & \$ & \$\$ & \$\$\$ \\
	Portability     & Excellent & Good & Poor & Poor \\
	{Volume\\coverage} & {Real-time\\3D volumes,\\improving} & 2D & {Large 3D\\volume} & {Large 3D \\volume} \\
	Contrast        & {Increasing\\(shear)} & Limited & Limited & {Slightly\\flexible} \\
	Intervention    & {Real-time\\3D increasing} & No\TblrNote{c} & No & Yes, limited \\
	Functional      & {Functional\\ultrasound} & No & No & fMRI \\
	\bottomrule
\end{talltblr}
\end{table}

Since 2004, medical imaging has been reported to have been performed more than 5 billion times \cite{Picano2004}, and later numbers from 2011 show a general doubling and in particular, a tenfold increase in ultrasound examinations between 2000 and 2011 \cite{Szabo_UltrasoundBook_2}. Recent data reveal that this trend of doubling has continued throughout the years 2010 to 2020 \cite{Winder2021}, and reveal that even though patient processes were disrupted during the global SARS-CoV-2 pandemic, the number of medical imaging examinations per 1000 patients still increased. The reasons for this and, particularly, why ultrasound has seen a significant increase in use, can be attributed to its high, but inconsistent, resolution, cost-effectiveness, portability, and real-time interventional imaging. The downside of ultrasound is its limited penetration and restrictions for use in certain body parts. When comparing soft tissue examinations, which ultrasound is limited to, both \gls{ct} and \gls{mri} can image the entire body with consistent resolution and contrast, but are more expensive and have poor portability due to the immense size of their hardware.

The cardiovascular system, which transports oxygen and nutrients to tissue, produces a complex flow pattern that causes velocity fluctuations. Several \gls{cvd} are also known to cause abnormal blood flow. In studies published by Center for Disease Control, a person dies from CVD every 34 seconds in the United States and complications from CVD caused 229 billion USD between 2017 and 2018 \cite{cdc_2022}. As mentioned above, ultrasound is a powerful tool for performing non-invasive imaging of the cardiovascular system \cite{JensenUltrasoundBook,Hansen_thesis}, and has no adverse risk to patients. Determining \gls{psd} of a received signal is a common way to estimate blood velocity. A processed image of \gls{psd} over time is commonly known as a sonogram, where changes in blood velocity over time can be seen.

\section{Literature review}
The aim of this project is to study the application of ultrasound in the context of blood flow measurements. Various scientific articles have been studied to gain knowledge of previous research \cite{Jensen_Analysis_PW_1996,Jansson_Estimation_Perfusion,Huang_Smartphone_2012,JanaSmartphone2020,DingPMUTs,Ding_PW_Pmut,Xu2007_Pulser,Matsuoka_Doppler_Rabbit,Fish_Ultrasonic,Williams2006,Winckler2012,Wang2016,Wang2019,Tsang2009,Govindan2016,Xu2007_Pulser,PICpulser}. In addition, textbooks \cite{JensenUltrasoundBook,ShungUltrasound_Book,Szabo_UltrasoundBook_2} have also been instrumental in forming a solid knowledge base for the thesis.

The delimitation of this work is done through a literature review of the field. Articles such as \cite{Jensen_Analysis_PW_1996,Wells1998,PWDesignParameters,Jansson_Estimation_Perfusion} are articles which outline the Doppler ultrasound analysis.

\section{Project scope}
The desire is to build upon the vast knowledge already gathered by prominent researchers in the field of ultrasound systems for blood velocity estimation. Finally, using the knowledge gained, we designed and implemented an electronic device capable of performing these measurements using a novel approach. The system used in this project is called an Ultrasound Doppler flow-meter. Ultrasound Doppler flow-meters can be used to measure the velocity of blood flow in the human body. This is commonly done to assess the health of blood vessels and to diagnose and monitor conditions such as arteriosclerosis (hardening of the arteries) and deep vein thrombosis (blood clots in the veins). To measure blood velocity with an ultrasound Doppler flow-meter, a handheld probe is placed on the skin over the area of interest, such as an artery or vein. The probe contains a transducer that emits high-frequency ultrasound waves and receives the reflected waves. The Doppler shift in the frequency of the reflected waves is caused by the movement of the blood cells, and it is proportional to the velocity of the blood flow. The probe is connected to a portable ultrasound machine, which processes the Doppler shift and displays the velocity of the blood flow on a screen. The machine can also produce a color-coded map of the blood flow, which allows the user to visualize the velocity of the blood at different points within the vessel. Ultrasound Doppler flow-meters are non-invasive and safe to use, and they provide a quick and easy way to measure blood velocity. However, they are not always accurate, especially in cases where there is a high degree of turbulence or when there are air bubbles or solid particles present in the blood. They are also limited in their ability to measure blood flow in small vessels or in deep tissues. The goals of the project are written in \cref{tab:specifications}.

\begin{table}[ht]
	\centering
	\caption{Project specification}
	\label{tab:specifications}
	\begin{tblr}[]{%
			%width=.9\textwidth,
			colspec = {l
			},
			row{1} = {guard, m, font=\small\bfseries},
			%vlines, hlines,
		}
		\toprule
		Project specification	\\
		\midrule
		Study and research ultrasound and its principles and applications	\\
		Design and implement a device for ultrasound blood velocity estimation	\\
		Investigate and test the device in an experimental setting		\\
		Validate results with commercial equipment 						\\
		Make quantifiable performance measurements on system			\\
		Write a technical report documenting the project work			\\ \bottomrule
	\end{tblr}
\end{table}

The project is conducted under the guidance of advisors from the affiliated institutions \Gls{dtu}, Department of Electrical Engineering, Department of Applied Mathematics and Computer Science, and \Gls{kaist} at the Brain/Bio Medical Microsystems Laboratory. The report is divided into five chapters, and the first part is an introduction to the project. The second chapter will focus on explaining the theory of the topic of the project. The third chapter focuses on the synthesis of a system for experimental testing. The fourth chapter explains the production of the hardware. The fifth chapter will explain the testing methodology performed on the hardware. Finally, additional documentation of testing, code, circuit diagrams, and laboratory setups can be found in the appendix.