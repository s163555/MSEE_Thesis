\chapter{Introduction} \label{cha:introduction}
%\DeclareSIUnit{\fps}{ \translate{frames per second} }
The progress of diagnostic imaging has advanced significantly during the \nth{20} century. As the cost of high speed computational systems has grown increasingly accessible, so has the use of of medical imaging become prominent. Potentially millions of people have been spared painful exploratory surgery through noninvasive diagnostic imaging. And thus, lives can be saved by early diagnosis and intervention through medical imaging. Advancement in scientific visualization have in turn generated more complex datasets of increased size and quality. Four major technologies used are ultrasound, X-ray, \gls{ct}, and \gls{mri}. Each of the technologies have distinct advantages and disadvantages in biomedical imaging, thus each are still relevant for modern medicine. \Cref{tab:imaging_modalities} contains a comparison and summary of the various fundamental diagnostic imaging modalities. 

\begin{table}[ht]
	\centering
	\begin{tabularx}{\textwidth}{@{}XXXXX@{}}
		\toprule
		\textbf{Modality} & \textbf{Ultrasound} & \textbf{X-ray} & \textbf{CT} & \textbf{MRI} \\ \midrule
		Topic              & Longitudinal, shear, mechanical properties              & Mean X-ray tissue absorbtion & Local tissue X-ray absorbtion & Biochemistry (\textit{T1} and \textit{T2})    \\
		Access             & Small windows adequate                                  & 2 sides needed               & Circumferential around body   & Circumferential around body \\
		Spatial resolution & Frequency and axially dependent, \qtyrange{0.2}{3}{\milli\meter} & $\sim \qty{1}{\milli \meter}$           & $\sim \qty{1}{\milli \meter}$            & $\sim \qty{1}{\milli \meter}$          \\
		Penetration     & Frequency dependent, \qtyrange{3}{25}{\centi\meter} & Excellent               & Excellent          & Excellent                                 \\
		Safety          & Excellent for > 50 years          & Ionizing radiation      & Ionizing radiation & Very good                                 \\
		Speed           & Real-time & Minutes & 20 minutes & Typical: 45 minutes, fastest: Real-time (\glsxtrshort{low-res}) \\
		Cost            & \$                                            & \$                      & \$\$                 & \$\$\$                                       \\
		Portability     & Excellent                                     & Good                    & Poor               & Poor                                      \\
		Volume coverage & Real-time 3D volumes, improving               & 2D                      & Large 3D volume    & Large 3D volume                           \\
		Contrast        & Increasing (shear)                            & Limited                 & Limited            & Slightly flexible                         \\
		Intervention    & Real-time 3D increasing                       & No, fluoroscopy limited & No                 & Yes, limited                              \\
		Functional      & Functional ultrasound                         & No                      & No                 & fMRI                                      \\ \bottomrule
	\end{tabularx}
	\caption{Comparison of Imaging Modalities \cite{Szabo_UltrasoundBook_2}}
	\label{tab:imaging_modalities}
\end{table}

Since medical imaging has been reportedly performed over 5 billion times as of 2004 \cite{Picano2004}, and later numbers from 2011 show a general doubling, and particularly a ten-fold increase in ultrasound examinations between year 2000 and 2011 \cite{Szabo_UltrasoundBook_2}. Recent data reveals that this trend of doubling has continued through the years 2010 to 2020 \cite{Winder2021}, and reveals that even though patient processes were disrupted during the global SARS-CoV-2 pandemic, the number of medical imaging examinations per 1000 patients, still increased. The reasons for this, and particularly why ultrasound has been notably increased in use, can be attributed to its high, but inconsistent, resolution, cost effectiveness, portability, and real-time interventional imaging. The downside of ultrasound is its limited penetration and restrictions for use in certain body parts. When just comparing soft-tissue examinations, which ultrasound is limited to, both \glsxtrshort{ct} and \glsxtrshort{mri} can image the entire body with consistent resolution and contrast, but are more expensive and has poor portability due to immense size of its hardware. 

The cardiovascular system, which transports oxygen and nutrients to tissue, produces a complex flow pattern that cause velocity fluctuations. Several cardiovascular diseases are also known to cause abnormal blood flow. As mentioned earlier, ultrasound is a powerful tool for conducting non-invasive imaging of the cardiovascular system \cite{JensenUltrasoundBook,Hansen_thesis}, and has no adverse risk to patients. Determining \gls{psd} of a received signal is a common way to estimate blood velocity. A processed image of the \glsxtrshort{psd} over time is commonly known as a sonogram, where changes in blood velocity over time can be seen. 

\section{Project scope}
The aim of this project is to study the application of ultrasound in the context of blood flow measurements. Various scientific articles have been studied to gain knowledge of prior research \cite{Jensen_Analysis_PW_1996,Jansson_Estimation_Perfusion,Huang_Smartphone_2012,JanaSmartphone2020,DingPMUTs,Ding_PW_Pmut}. The desire is to build upon the vast knowledge already gathered by prominent researchers in the field of ultrasound systems for blood velocity estimation. Finally, using the knowledge gained, to design and implement a device capable of performing these measurements using a novel approach.

\subsection{Learning objectives}
See below for an outline of the project activities
\begin{table}[ht!]
	\centering
	\begin{tabular}{@{}l@{}}
		\toprule
		\textbf{Project specification}									\\ \midrule
		Learn a class-D amplifier topology, calculate component values	\\
		Understand and design a self-oscillating modulator amplifier	\\
		Investigate and test open loop output filter					\\
		Investigate and test closed loop output filter					\\
		Investigate output filter parasitic elements affects control loop\\
		Make quantifiable performance measurements on system			\\
		Write a technical report documenting the project work			\\ \bottomrule
	\end{tabular}
	\caption{Project specification table}
	\label{tab:specifications}
\end{table}