\chapter{Production}

\begin{figure}[H]
	\centering
	\begin{circuitikz}[american voltages]
		\draw
		(0,0) to [short, *-] (6,0)
		to [V, l_=$\mathrm{j}{\omega}_m \underline{\psi}^s_R$] (6,2) 
		to [R, l_=$R_R$] (6,4) 
		to [short, i_=$\underline{i}^s_R$] (5,4) 
		(0,0) to [open, v^>=$\underline{u}^s_s$] (0,4) 
		to [short, *- ,i=$\underline{i}^s_s$] (1,4) 
		to [R, l=$R_s$] (3,4)
		to [L, l=$L_{\sigma}$] (5,4) 
		to [short, i_=$\underline{i}^s_M$] (5,3) 
		to [L, l_=$L_M$] (5,0); 
	\end{circuitikz}
	\caption{The nodes short, V, R and L are presented here, but there a lot more}
	\label{fig:circuitikz}
\end{figure}

\section{Listings (code)}

\Cref{lst:helloworld} is a nicely formatted block of code. A listing will automatically continue on the next page if it encounters a page break. Many different programming languages can be highlighted. Check the \texttt{listings} package documentation for a list of supported programming languages. 

\begin{listing}[htbp]
\begin{mintedc}
#include <stdio.h>
int main() 
{
	printf("Hello, World!"); /*printf() outputs the quoted string*/
	if (n == 0 || n == 1){    
		return n;        
	}        
	j = 0;    
	for (i = 0; i < n-1; i++){      
		if (arr[i] != arr[i+1]){        
			arr[j] = arr[i];       
			j++;      
		}    
	}      
	arr[j++] = arr[n-1];
	return 0;
}
\end{mintedc}
	\caption{Hello world in C}
	\label{lst:helloworld}
\end{listing}



