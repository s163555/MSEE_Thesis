\chapter{Conclusion}
In conclusion, this project has successfully gone through thorough research, data collection, and analysis, and have obtained a comprehensive understanding of the subject matter. Through a literature review, key gaps in the research have been identified. This formed the basis of a design approach to the ultrasound system that was followed throughout the project. This project aimed to design, build, and test an ultrasound system for blood velocity estimation that can be miniaturised. The implementation of all the sub-components in the system was simulated and/or tested thoroughly to verify the function. Moreover, some of the combined testing revealed some implementation changes necessary in the power stage output. Despite the numerous achievements and contributions made throughout this project, it is important to acknowledge its limitations. Factors such as time constraints, limited resources, and external factors beyond our control may have impacted the scope and depth of our study. However, the findings and insights presented in this report should provide a solid foundation for future research and further exploration of the topic. Due to time limitations, experiments were focused on module testing each part of the system independently. It is therefore difficult to compare it with prior research directly. The system was designed around a control system based on the Zynq 7020 SoC, which contains both the configured ultrasound pulser logic controller and the Python-based programmable interface. The transmitter, which includes the ultrasound pulser and the power stage, was tested with the switch, and while the transmitting was successful, a received reflected signal could not be found. It is believed that the received switch was sinking into the on-board load of the power stage.  On the receiver, which includes the bandpass filter, preamplifier, quadrature demodulator, sample and hold circuit, and active filter, were also verified functioning as intended in isolated experiments. The implementation of the analogue frontend can result in a more compact commercial product. However, more work is required before a working single-device prototype can be build and evaluated.
\section{Future Work}
The system was fully validated, although in module testing. TThe absence of eflected signals, when using the  power stage, should be further investigated. In the future, it is desired to perform a complete system test and evaluate the performance using the physiological simulator. After a successful experiment, a single PCB layout can be completed from the CAD work already done for the documentation in Altium Designer. Another future work would be further implementing a continuous data acquisition with a spectral image, thus fully displaying the sonographic data in the DSP. 