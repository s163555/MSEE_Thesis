\chapter*{Reading comprehension} \label{cha:report_structure}
\addcontentsline{toc}{chapter}{Reading comprehension}
This section of the report will explain to the reader how to reference this document and explain the fundamental structure of the project as well as the report. Throughout the report, the reader will be assumed to be knowledgeable of basic circuit analysis and familiar with standard abbreviations typically used in electrical engineering. If not, readers can refer to the denotation section at the beginning of the report. It is assumed that the reader has a basic knowledge on the science of physics, electrical engineering, computing, and circuit analysis.

Please refer to \hyperref[cha:acronyms]{Acronyms}, \hyperref[cha:glossary]{Glossary}, and \hyperref[cha:nomenclature]{Nomenclature} pages for explanations to terms found within the report.

%Furthermore, as a notation convention, large-signal \gls{dc} first and quantities are denoted by uppercase letters with uppercase subscripts. Small-signal quantities are denoted using lowercase letters with lowercase subscripts. Quantities composed of both large-signal and small-signal elements are denoted using lowercase letters and uppercase subscripts.

\section*{Sources}
Calculus expressions present in the report will typically have a reference explaining their origin. All references are prominently displayed with square brackets and a number, directing to the appendix in the last section of the report.

